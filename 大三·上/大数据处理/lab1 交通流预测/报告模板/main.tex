\documentclass[4pt]{article}
\usepackage[utf8]{inputenc}
\usepackage{amsmath,amssymb}
\usepackage{graphicx}
\usepackage{hyperref}
\usepackage{geometry}
\usepackage{enumerate}
\usepackage[fontset=ubuntu]{ctex}
\usepackage{caption}
\usepackage{tabularx}
\usepackage{float}
\usepackage{longtable}
\usepackage{subcaption}
\usepackage{multirow}
\usepackage{booktabs}
\usepackage{algpseudocode}
\usepackage{pythonhighlight}
\usepackage{natbib}
\bibliographystyle{unsrt}
\geometry{a4paper,scale=0.9}

\title{实验是个好东西}
\author{xxxx}
\date{}

\begin{document}
\vspace{-10cm}
\maketitle

\section{交通流预测问题描述}

随着城市化的快速发展,交通拥堵问题日益严重,给人们的出行和城市的可持续发展带来了很大的挑战。交通流预测是智能交通系统中的一个战略性需求,旨在缓解交通压力、协调运营时间表和规划未来建设需要对交通流进行精准预测,从而为交通管理和规划提供有力的数据支持。

\section{预测模型的介绍和分析} 

本文采用的模型整体是一个时间序列预测框架\cite{teach2021},特别是针对交通流量数据。该框架集成了多种预测模型,包括历史平均(HA)、线性回归(LR)、长短时记忆网络(LSTM)、门控循环单元(GRU)、图卷积神经网络(STGCN)、门控图波浪网络(GWN)以及自回归积分滑动平均模型(ARIMA)。
    
\begin{table}[H]
    \centering
    \small % Reduce font size
    \begin{tabularx}{\textwidth}{|X|X|X|}
        \hline
        \textbf{模型名称} & \textbf{介绍} & \textbf{特点分析} \\
        \hline
        HA(历史平均) & 简单地使用过去一段时间的数据的平均值作为未来的预测。 & 非常简单和快速,但可能不适用于复杂和非线性的时间序列。 \\
        \hline
        GRU(门控循环单元) & 是一种循环神经网络(RNN)的变体,用于序列数据的模型化。 & 比传统的 RNN 更高效,在处理长序列时也更稳定,但可能需要更多的数据进行训练。 \\
        \hline
        LSTM(长短时记忆) & 另一种 RNN 变体,特别设计用于解决长序列学习中的长期依赖问题。 & 非常适用于长序列和复杂模式的数据,但计算成本可能相对较高。 \\
        \hline
        STGCN(时空图卷积网络) & 一种用于时空数据(例如交通网络)的图卷积网络。 & 能够同时考虑时间和空间的依赖关系,但需要大量的计算资源。 \\
        \hline
        GWN(图波形网络) & 一种用于图结构数据的深度学习模型,特别是用于交通流量预测。 & 能够捕获复杂的时空依赖关系,但同样需要大量的计算资源。 \\
        \hline
        ARIMA(自回归积分滑动平均模型) & 一种统计模型,用于时间序列数据的预测。 & 适用于多种类型的时间序列数据,但可能需要专门的参数调整。 \\
        \hline
    \end{tabularx}
    \caption{预测模型的介绍和特点分析}
    \label{tab:model_analysis}
    \normalsize % Reset to normal font size
\end{table}

\section{实验数据集介绍}


\subsection{Metr-LA}

这个数据集源自美国洛杉矶的传感器网络。数据通常包括交通速度、流量和其他可能的特征。广泛用于交通流量预测、异常检测等。本人在实验中选择的就是此数据集。

\subsection{PeMSD4}

这个数据集是从加利福尼亚州的 PeMS(Performance Measurement System)系统中获得的。通常包括不同地点、不同时间的交通流量数据。这个数据集主要用于交通流量预测,也可能用于交通管理系统的其他应用。

\subsection{PeMSD8}

与 PeMSD4 类似,这也是从 PeMS 系统中获得的数据。PeMSD8 通常涵盖更多的传感器和更长的时间范围。除了交通流量预测,这个数据集也可以用于研究交通模式、拥堵模式等。

\section{评价指标介绍}

\subsection{平均绝对误差(Mean Absolute Error, MAE)}

MAE 是预测值与实际值之差的绝对值的平均。该指标容易理解和解释,单位与数据相同。

\[
\text{MAE} = \frac{1}{n} \sum_{i=1}^{n} \left| y_{\text{true}, i} - y_{\text{pred}, i} \right|
\]

\subsection{均方根误差(Root Mean Square Error, RMSE)}

要计算均方根误差 RMSE, 首先需要计算均方误差 MSE ,它是预测值与实际值之差的平方的平均。相对于 MAE,MSE 更重视较大的误差。

\[
\text{MSE} = \frac{1}{n} \sum_{i=1}^{n} \left( y_{\text{true}, i} - y_{\text{pred}, i} \right)^2
\]

而 RMSE 是 MSE 的平方根,使得误差的单位与原数据相同。它给予较大误差更高的权重。

\[
\text{RMSE} = \sqrt{\text{MSE}}
\]

本实验采用的这两种指标是用于回归问题和时间序列预测中最常用和最广泛接受的指标。各有优缺点:

MAE: 计算简单,直观,容易解释。它给出了模型预测与真实值之间的平均绝对偏差。但它不考虑预测错误的方向,只关心大小。

RMSE: 同样是一个常用的指标,用于量化预测误差的平均大小。由于平方误差在大误差上会被放大,RMSE 对大误差更敏感。

本实验将这两个指标一起使用,以提供关于模型性能的更全面视图。MAE 更关注于整体误差级别,而 RMSE 更侧重于大误差的影响。如果模型对所有情况的预测都很均匀,则 MAE 和 RMSE 通常会接近;如果模型对某些情况的预测有大的偏差,则 RMSE 通常会明显高于 MAE。

\section{实验结果及分析}

模型的原始参数如图1中代码所示:



\begin{figure}[h]
    \centering
    \begin{subfigure}{0.423\textwidth}
        \includegraphics[width=\textwidth]{figures/原始数据/HA_LR_predict.png}
        \caption{HA\&LR}
    \end{subfigure}
    \begin{subfigure}{0.45\textwidth}
        \includegraphics[width=\textwidth]{figures/原始数据/STGCN_predict.png}
        \caption{STGCN}
    \end{subfigure}
    \begin{subfigure}{0.45\textwidth}
        \includegraphics[width=\textwidth]{figures/原始数据/GWN_GRU_predict.png}
        \caption{GWN\&GRU}
    \end{subfigure}
    \begin{subfigure}{0.45\textwidth}
        \includegraphics[width=\textwidth]{figures/原始数据/predict_results.png}
        \caption{All}
    \end{subfigure}
    \caption{原始参数下各模型预测结果}
\end{figure}


%长表格
\setlength{\LTleft}{2cm}  % 负值将表格向左移动,正值向右
\large      % 调整字体大小
% 只需要改上面这两个属性即可调整
\begin{longtable}{ccccccc} % 使用的是longtable宏包
\caption{三组实验结果} \\
\toprule
\textbf{Model} & \multicolumn{2}{c}{\textbf{原参数}} & \multicolumn{2}{c}{\textbf{修改参数}} & \multicolumn{2}{c}{\textbf{修改训练集比例}} \\
\cmidrule(r){2-3} \cmidrule(r){4-5} \cmidrule(r){6-7}
& mae & rmse & mae & rmse & mae & rmse \\
\midrule
\endfirsthead
\multicolumn{7}{c}{(Continued)} \\
\midrule
\textbf{Model} & \multicolumn{2}{c}{\textbf{原参数}} & \multicolumn{2}{c}{\textbf{修改参数}} & \multicolumn{2}{c}{\textbf{修改训练集比例}} \\
\cmidrule(r){2-3} \cmidrule(r){4-5} \cmidrule(r){6-7}
& mae & rmse & mae & rmse & mae & rmse \\
\midrule
\endhead
\bottomrule
\endlastfoot
HA & 0.0671632 & 0.152194 & 0.079333 & 0.110885 & 0.0835541 & 0.183408 \\
\end{longtable}


针对METR-LA数据集,对不同的模型进行了评估。METR-LA是一个交通流量数据集,因此,选择的模型需要能够捕捉时间序列数据的动态性和模式。

\begin{enumerate}
  \item \textbf{默认参数实验}:
    \begin{itemize}
      \item 最佳MAE: GWN模型 (0.)
      \item 最佳RMSE: ARIMA模型 (0.)
    \end{itemize}

  \item \textbf{修改参数实验}:
    \begin{itemize}
      \item 最佳MAE: GWN模型 (0.)
      \item 最佳RMSE: GWN模型 (0.)
    \end{itemize}

\end{enumerate}

\textbf{模型分析}:
\begin{itemize}
  \item \textbf{GWN模型}:METR-LA数据集的交通流量。

  \item \textbf{LSTM模型}:相比于其他模型,
\end{itemize}

\section{结论与下一步工作(创新点)}

\subsection{结论}

在所有的实验中,GWN和ARIMA模型对于M
\subsection{下一步工作(创新点)}

尝试使用,代码和模型原理详见附录,测试效果如下图所示:

注:附录见下页
\newpage
%附录
\appendix
\section{xx算法预测}
 \begin{python}
import pandas as pd
import matplotlib.pyplot as plt
import datetime

# 创建并拟合模型
df.columns = ['ts'] + df.columns[1:].tolist()
df.head()

df.columns = ['ts'] + df.columns[1:].tolist()
df.head()

# 计算指标
rae = mean_absolute_error(actual, predicted)
rmse = np.sqrt(mean_squared_error(actual, predicted))

print(f"RAE: {rae}")
print(f"RMSE: {rmse}")
\end{python}

\section*{模型介绍}

\subsection*{1. 趋势}

趋势部分可以进一步分为线性趋势和饱和增长趋势。

\bibliography{references}     % 'references' corresponds to the filename `references.bib`.

\end{document}
