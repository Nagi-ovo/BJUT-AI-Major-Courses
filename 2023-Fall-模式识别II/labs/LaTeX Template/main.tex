\documentclass[12pt]{article}
\usepackage[utf8]{inputenc}
\usepackage{amsmath,amssymb}
\usepackage{graphicx}
\usepackage{hyperref}
\usepackage{geometry}
\usepackage{enumerate}
\usepackage[fontset=ubuntu]{ctex}
\usepackage{caption}
\usepackage{tabularx}
\usepackage{float}
\usepackage{longtable}
\usepackage{subcaption}
\usepackage{multirow}
\usepackage{booktabs}
\usepackage{algpseudocode}
\usepackage{pythonhighlight}
\usepackage{natbib}
\usepackage{minted,color}
\setminted{
  framesep=2mm,
  bgcolor=bg,
  linenos
}
\definecolor{bg}{rgb}{0.95,0.95,0.95}
\usemintedstyle{tango}

\bibliographystyle{unsrt}
\geometry{a4paper,scale=0.9}


\begin{document}

\begin{titlepage}
    \centering
    \vspace*{2cm}
    
    \Huge
    \textbf{实验是个好东西}
        
    \vspace{2cm}
        
    \Large
    姓名:  \\ \vspace{0.5cm}
    学号: \\ \vspace{0.5cm}
    专业: 人工智能\\ \vspace{0.5cm}
    指导老师: \\

    
    \vfill
    \normalsize
    
    \begin{center}
    \hspace*{-1.1cm}
    \begin{tabular}{|p{8cm}|p{8cm}|}
    \hline
    \multicolumn{2}{|c|}{{\large \textbf{报告分:(100 points)}}} \\
    \hline
    \textbf{Attendance (5 points)} & \textbf{Results (10 points)} \\
    \hline
    1.实验课时签到成功 (5) & 1.给出算法正确和可运行程序 (10) \\
    2.实验课时没有签到成功 (0) & 2.程序算法原理有部分理解错误 (5) \\
    & 3.程序算法原理和程序都有完全错误 (0) \\
    \hline
    \textbf{Content of Report (55 points)} & \textbf{Format in Report (30 points)} \\
    \hline
    1.要求点全部高质量完成(55) & 1.图表、引用等符合学术规范 (15)、语言清晰,逻辑明确和表述清晰完整(15) \\
    2.要求点完成但有错误(给分点减半) & 2.图表、引用不符合学术规范(0) \\
    3.要求点没有完成(扣相应给分点的分) & 3.语言清晰,逻辑明确和表述清晰完整(0) \\
    \hline
    \end{tabular}
    \end{center}

    \begin{center}
    \begin{tabular}{|l|}
    \hline
    最终本次实验得分成绩= 报告分 *(时间或算法)性能排序因子 \\
    \hline
    (时间或算法)性能排序因子为: \\
    · top1 - 3,100 \\
    · Top4 - 10,95 \\
    · Top11 - 15,90 \\
    · Top16 - 最后,80 \\
    性能可以并列第一。 \\
    \hline
    \end{tabular}
    \end{center}


\end{titlepage}

\vspace{-10cm}

\section{问题描述}
实验要求使用朴素贝叶斯分类器来对MNIST数据集进行分类。MNIST数据集是手写数字的数据集,其中每个样本都是一个28x28像素的图像,对应0到9的数字标签。

\section{问题的本质和分析}
朴素贝叶斯方法对手写数字图像进行分类,朴素贝叶斯分类器天真地假设所有特征(这里是像素)之间是条件独立的。

\section{解决问题的思路和结果预测}
首先进行数据的预处理,将像素值二值化,使其变为0或1。然后我们使用训练集来计算先验概率和特征的条件概率。最后,我们使用这些概率来预测测试集的类标签。

\begin{figure}[H]
    \centering
    \hspace*{-1.4cm}
    \includegraphics[width=0.5\textwidth]{figures/0.png}
    \caption{简单了解数据}
    \label{fig:your_label}
\end{figure}

考虑到MNIST数据集的复杂性,不难猜到这种简单的方法可能不会达到最先进的结果,但应该可以获得令人接受的准确率。

\section{解决问题中遇到的难点}
1. 数据量大,需要高效的计算方法。

2. 二值化可能丢失一些图像的重要信息。

3. 对于图像数据,特征间的条件独立假设可能不太合理。

4. 单纯的朴素贝叶斯肯定有不够高的上限。

\section{关键代码的实现与解释}

\subsection{先验概率的计算}

公式:
\[ P(y) \]

\begin{minted}{python}
self.class_log_prior_ = np.log(np.bincount(y) / len(y))
\end{minted}
计算每个类$y$的先验概率,并将结果保存为对数值。`np.bincount(y)`计算每个类在数据集中出现的次数,然后除以总样本数得到每个类的先验概率。

\subsection{计算特征的条件概率}


\subsection{预测}



选择具有最高后验概率的类作为预测标签。

\section{结果分析与改进}

\subsection{分析}
使用贝叶斯分类器。

\subsection{改进的动机}

使用贝叶斯分类器对MNIST数据集进行分类时,准确率只达到了约82.89\%。

\subsection{依据}
\begin{enumerate}
    \item \textbf{xxx}:
    
    \item \textbf{xxxx}:
\end{enumerate}

\subsection{算法设计}
结合先前的朴素贝叶斯模型以得到更好的预测结果。

\subsection{预期结果}
加上原先的方法肯定能够得到更高的正确率。

\section{改进的代码和结果分析}
\subsection{代码及运行结果}

完整代码详见附件:

\begin{minted}{python}
# 可视化降维后的数据
visualize_tsne(transformed_data, targets)
\end{minted}

运行结果见 [图1] ,从图中可以得到如下信息:

\begin{enumerate}
    \item \textbf{类别聚集}:不同颜色。
    
    \item \textbf{区分明确}:展示了手写数字降维后提取出了独特性。
    
    \item \textbf{部分重叠}:簇的部分边缘存在重叠。
\end{enumerate}


\subsection{结论}

多参数(如困惑度)的影响,所以可视化的结果并不总是与分类器的实际性能一致。

如果想得到预测正确率,还需要加上前面定义的朴素贝叶斯来进行模型的训练和验证。

\begin{enumerate}
    \item \textbf{数据预处理}:首先,

    \item \textbf{模型训练}:使用朴素贝叶斯分类器对数据进行分类。

    \item \textbf{结果对比}:与原始数据上的 83\% 准确率相比,降维后的数据使得准确率显著提升至 96.1\%。
\end{enumerate}

结论:我成功地提高了 MNIST 手写数字分类的准确性,验证了这种策略的高效性。

\bibliography{references}     % 'references' corresponds to the filename `references.bib`.

\clearpage  % 使用 \clearpage 确保所有浮动内容(例如图表)都被处理了

\vspace*{\fill}
\begin{center}
    \textbf{注:附录见下页}
\end{center}
\vspace*{\fill}

\clearpage  % 再次使用 \clearpage 以确保注释部分与附录分离

\newpage
%附录
\appendix

\section*{完整代码:lab.ipynb}
\begin{minted}{python}
import matplotlib.pyplot as plt
import numpy as np
import pandas as pd
import seaborn as sns
from sklearn import datasets
from sklearn import manifold
from sklearn.metrics import accuracy_score

# 获取数据
data = datasets.fetch_openml('mnist_784', version=1, return_X_y=True)
pixel_values, targets = data
targets = targets.astype(int)
pixel_array = pixel_values.to_numpy()
single_image = pixel_array[1, :].reshape(28, 28)

# 展示图像
plt.imshow(single_image, cmap='gray')
plt.show()
\end{minted}

\begin{figure}[H]
    \centering
    \hspace*{-1.4cm}
    \includegraphics[width=0.5\textwidth]{figures/0.png}
    \caption{简单了解数据}
    \label{fig:your_label}
\end{figure}

\begin{minted}{python}
# 使用自己定义的朴素贝叶斯训练模型
clf = GaussianNaiveBayes()
clf.fit(X_train, y_train)

# 在测试集上进行预测
y_pred = clf.predict(X_test)

# 计算准确率
accuracy = accuracy_score(y_test, y_pred)
print(f"Accuracy: {accuracy*100:.2f}%")
\end{minted}

最终结果:

Accuracy: 00.10\%
\end{document}
